\documentclass[a4paper,12pt,oneside]{book} % nie: report!


% pakiety
\usepackage{polski} % lepiej to zamiast babel!
\usepackage[utf8]{inputenc} % w razie kłopotów spróbować: \usepackage[utf8x]{inputenc}
\usepackage{fancyhdr} % nagłówki i stopki
\usepackage{indentfirst} % WAŻNE, MA BYĆ!
\usepackage[pdftex]{graphicx} % to do wstawiania rysunków
\usepackage{amsmath} % to do dodatkowych symboli, przydatne
\usepackage[pdftex,
            left=1in,right=1in,
            top=1in,bottom=1in]{geometry} % marginsy
\usepackage{amssymb} % to też do dodatkowych symboli, też przydatne
\usepackage{pdfpages}
\usepackage{lipsum}
\usepackage{multirow}
\usepackage{listings}
\usepackage{caption}
\usepackage{booktabs}
\graphicspath{ {./images/} }
\DeclareCaptionType{code}[Listing][Spis listingów] 

% definicje nagłówków i stopek
\pagestyle{fancy}
\renewcommand{\chaptermark}[1]{\markboth{#1}{}}
\renewcommand{\sectionmark}[1]{\markright{\thesection\ #1}}
\fancyhf{}
\fancyhead[LE,RO]{\footnotesize\bfseries\thepage}
\fancyhead[LO]{\footnotesize\rightmark}
\fancyhead[RE]{\footnotesize\leftmark}
\renewcommand{\headrulewidth}{0.5pt}
\renewcommand{\footrulewidth}{0pt}
\addtolength{\headheight}{1.5pt}
\fancypagestyle{plain}{\fancyhead{}\cfoot{\footnotesize\bfseries\thepage}\renewcommand{\headrulewidth}{0pt}}


% interlinia
\linespread{1.25}


% treść
\begin{document}
\sloppy
\thispagestyle{empty}
\includepdf{stronatytulowa}
\newpage{}

\thispagestyle{empty}
\newpage{}

\tableofcontents{}

\chapter*{Wstęp}
\addcontentsline{toc}{chapter}{Wstęp}
\label{Wstep}
%Mega ogólne sprawy dotyczące tego co chce zrobić
\lipsum[1]

\chapter*{Cel i zakres pracy}
\addcontentsline{toc}{chapter}{Cel i zakres pracy}
\label{Cel i zakres pracy}
%celem pracy jest napisanie programu bla bla
\lipsum[1]

\chapter{Systemy rekomendacji}
\lipsum[1]\cite{test1}

\section{Historia}
\lipsum[1]\cite{test2}

\lipsum[1]\cite{test2}

\lipsum[1]\cite{test2}


\section{Architektura}
\lipsum[1]\cite{test4}

\section{Wady i zalety}
\lipsum[1]\cite{test5}

\begin{figure}[h]
\centering
\includegraphics[scale=0.8]{obelix.png}
\caption{Obelix [Google Images]}
\end{figure}

\lipsum[1]\cite{test5}


\chapter{Algorytmy rankingowe}

\section{Section1}

\lipsum[1]

\subsection{Subs1}
\lipsum[1]

\subsection{Subs2}
\lipsum[1]

\section{Section2}
\lipsum[1]

\section{Section3}
\lipsum[1]

\subsection{Subs1}
\lipsum[1]

\subsection{Subs2}
\lipsum[1]

\chapter{Ontologie i grafowe bazy danych}
\lipsum[1]

\chapter{Opracowanie architektury i wybór narzędzi}
\lipsum[1]

\chapter{Implementacja systemu}
\lipsum[1]

\section{Ontologie}
\lipsum[1]

\section{Dane}
\lipsum[1]

\section{Algorytm rankingowy}
\lipsum[1]

\section{Uproszczona aplikacja}
\lipsum[1]

\section{Analiza błędów}
\lipsum[1]

\chapter{Podsumowanie}


\listoftables{} % jeśli są tabele
\addcontentsline{toc}{chapter}{Spis tabel}

\listoffigures{} % jeśli są rysunki
\addcontentsline{toc}{chapter}{Spis rysunków}

\listofcodes{}
\addcontentsline{toc}{chapter}{Spis listingów}

\addcontentsline{toc}{chapter}{Bibliografia}
\bibliographystyle{ieeetr}
\bibliography{bibliography/cite}


\end{document}
